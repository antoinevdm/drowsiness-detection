\documentclass{beamer}

\usepackage{graphicx}
\usetheme{metropolis}           % Use metropolis theme

\title{Computer vision used for alertness detection}
\date{\today}
\author{Antoine Vander Meiren \& Lorenzo Riga \& Gaetano Giordano}
\institute{ECAM Brussels}

\begin{document}
\maketitle
\begin{frame}{Driver alertness}
Two main concerns:
	\begin{itemize}
        \item Driver drowsiness \footnote{source: CDC (US, 2013) }:
    \begin{itemize}
        \item 72,000 crashes
        \item 44,000 injuries
        \item 800 deaths
    \end{itemize}
    \item Use of self phone while driving \footnote{source: WHO (US, 2005-2007)} \\
        \hspace{0.27cm} \textit{11\% of crashes}
	\end{itemize}
\end{frame}

\begin{frame}{Driver alertness}
A real need to combat driver distraction \\
through the growth of embedded technologies using:
    \begin{itemize}
    \item Computer vision
    \item Neural networks
    \end{itemize}
\end{frame}

\section{Artificial Intelligence}
\begin{frame}{Computer vision}
	\begin{itemize}
	\item Recreating the human eye \\
        \hspace{0.27cm} \textit{Modern CCD sensors more sensitive than a human eye}
    \item Recreating the human brain to interpret the output \\
        \hspace{0.27cm} \textit{Underlying software at the heart of computer vision}
	\end{itemize}
\end{frame}

\begin{frame}{Image interpretation}
Our brain is built with vision in mind \\
Sense for which it allocates the most processing power \\
\begin{itemize}
\item groups of neurons excite each other when contrast or motion is detected
\item higher level neurons aggregate this in meta-patterns i.e. A circle moving upwards
\item other groups recognize colors
\end{itemize}
The brains paints a mental picture
\end{frame}

\begin{frame}{Computer vision: Top-down vs. Bottoms-up}
    \begin{itemize}
        \item Top-down approach: \textbf{Naive approach} \\
        \item Tell the computer: this is what a book looks like \\
            \hspace{0.27cm} \textit{What if the book is on its side ? or the apple cut in half?} \\
        \item Requires to store pictures of every object in every configuration and from every angle
    \end{itemize}
\end{frame}

\begin{frame}{Computer vision: Top-down vs. Bottoms-up}
    \begin{itemize}
        \item Bottom-up approach: \textbf{Smarter approach} \\
        \item Apply transformation to objects to detect edges \\
            \hspace{0.27cm}\textit{Through math and statistics match to a trained dataset} \\
        \item Similar to what the brain achieves

\end{itemize}
    \end{frame}

\begin{frame}{Machine learning}
    \begin{itemize}
        \item Techniques to give computer systems the ability to ``learn''
    \end{itemize}
    \vspace{1cm}

    ``Learning is the human process that allows us to acquire the skills necessary to adapt to the multitude of situations we encounter.'' [Japkowicz and Shah (2011)]

\end{frame}

\begin{frame}{Machine learning: Neural networks}
    \begin{itemize}
        \item Learning systems inspired by the human brain
        \item Cluster of neurons linked together \\
            \hspace{0.27cm}\textit{Optimized by adjusting links' weights}
        \item Supervised learning with a labeled dataset
    \end{itemize}
\end{frame}

\section{Tools}
\begin{frame}{OpenCV}
    \begin{itemize}
        \item Open source library with thousands of algorithms for:
        \begin{itemize}
            \item Detect and recognize faces
            \item Identify objects
            \item Track movements
            \item Etc...
        \end{itemize}
        \item Strong focus on real-time applications
        \item Free for use under the open-source BSD license \footnote{License imposing minimal restrictions on the use and redistribution of covered software}
        \item Supports deep learning frameworks \\
            \hspace{0.27cm}\textit{TensorFlow, Torch/PyTorch and Caffe}
    \end{itemize}
\end{frame}

\begin{frame}{Caffe}
    \begin{itemize}
        \item CAFFE: Convolutional Architecture for Fast Feature Embedding  \\
            \hspace{0.27cm}\textit}


    \end{itemize}
\end{frame}

\section{Project}
\begin{frame}{Features}
    Object du projet, quoi detecter
\end{frame}
\begin{frame}{Integration of tools}
    Comment utliser les tools
\end{frame}
\begin{frame}{How does it works}
    Comment son faites les comparaisons (bout de code ?)
\end{frame}
\begin{frame}{Demo}
    Lancer la vider
\end{frame}
\end{document}
