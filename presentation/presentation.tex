\documentclass{beamer}

\usepackage{graphicx}
\usepackage{multimedia}
\usetheme{metropolis}           % Use metropolis theme

\title{Computer vision used for alertness detection}
\author{\textbf{Giordano} Gaetano \\ \textbf{Riga} Lorenzo \\ \textbf{Vander Meiren} Antoine \\}
\date{June 5, 2018}
\institute{ECAM Brussels}

\begin{document}
\maketitle
\begin{frame}{Driver alertness}
Two main concerns:
	\begin{itemize}
        \item Driver drowsiness \footnote{source: CDC (US, 2013) }:
    \begin{itemize}
        \item 72,000 crashes
        \item 44,000 injuries
        \item 800 deaths
    \end{itemize}
    \item Use of self phone while driving \footnote{source: WHO (US, 2005-2007)} \\
        \hspace{0.27cm} \textit{11\% of crashes}
	\end{itemize}
\end{frame}

\begin{frame}{Driver alertness}
A real need to combat driver distraction \\
through the growth of embedded technologies using:
    \begin{itemize}
    \item Computer vision
    \item Neural networks
    \end{itemize}
\end{frame}

\section{Artificial Intelligence}
\begin{frame}{Computer vision}
	\begin{itemize}
	\item Recreating the human eye \\
        \hspace{0.27cm} \textit{Modern CCD sensors more sensitive than a human eye}
    \item Recreating the human brain to interpret the output \\
        \hspace{0.27cm} \textit{Underlying software at the heart of computer vision}
	\end{itemize}
\end{frame}

\begin{frame}{Image interpretation}
Our brain is built with vision in mind \\
Sense for which it allocates the most processing power \\
\begin{itemize}
\item groups of neurons excite each other when contrast or motion is detected
\item higher level neurons aggregate this in meta-patterns i.e. A circle moving upwards
\item other groups recognize colors
\end{itemize}
The brains paints a mental picture
\end{frame}

\begin{frame}{Computer vision: Top-down vs. Bottoms-up (1)}
    \begin{itemize}
        \item Top-down approach: \textbf{Naive approach} \\
        \item Tell the computer: this is what a book looks like \\
            \hspace{0.27cm} \textit{What if the book is on its side?} \\
        \item Requires to store pictures of every object in every configuration and from every angle
    \end{itemize}
\end{frame}

\begin{frame}{Computer vision: Top-down vs. Bottoms-up (2)}
    \begin{itemize}
        \item Bottom-up approach: \textbf{Smarter approach} \\
        \item Apply transformation to objects to detect edges \\
            \hspace{0.27cm}\textit{Through math and statistics match to a trained dataset} \\
        \item Similar to what the brain achieves

\end{itemize}
    \end{frame}

\begin{frame}{Machine learning}
    \begin{itemize}
        \item Techniques to give computer systems the ability to ``learn''
    \end{itemize}
    \vspace{1cm}

    ``Learning is the human process that allows us to acquire the skills necessary to adapt to the multitude of situations we encounter.'' [Japkowicz and Shah (2011)]

\end{frame}

\begin{frame}{Machine learning: Neural networks}
    \begin{itemize}
        \item Learning systems inspired by the human brain
        \item Cluster of neurons linked together \\
            \hspace{0.27cm}\textit{Optimized by adjusting links' weights}
        \item Supervised learning with a labeled dataset
    \end{itemize}
\end{frame}

\section{Tools}
\begin{frame}{OpenCV}
    \begin{itemize}
        \item Open source library with thousands of algorithms for:
        \begin{itemize}
            \item Detect and recognize faces
            \item Identify objects
            \item Track movements
            \item Etc...
        \end{itemize}
        \item Strong focus on real-time applications
        \item Free for use under the open-source BSD license \footnote{License imposing minimal restrictions on the use and redistribution of covered software}
        \item Supports deep learning frameworks \\
            \hspace{0.27cm}\textit{TensorFlow, Torch/PyTorch and Caffe}
    \end{itemize}
\end{frame}

\begin{frame}{Caffe}
    \begin{itemize}
        \item CAFFE: Convolutional Architecture for Fast Feature Embedding  \\
            \hspace{0.27cm}\textit{Deep learning framework}
        \item Open source under BSD license
        \item Written in C++
            \hspace{0.27cm}\textit{with a Python interface}
        \item Lots of pre-trained model available for free
            \hspace{0.27cm}\textit{https://github.com/BVLC/caffe}
    \end{itemize}
\end{frame}

\section{Project}
\begin{frame}{Features (1)}
    \begin{itemize}
        \item Object detection:
            \begin{itemize}
                \item Face
                \item Eyes
                \item Cellphone
            \end{itemize}
        \item Alert on abnormal behavior
    \end{itemize}
\end{frame}

\begin{frame}{Features (2)}
    \begin{itemize}
        \item Face and eyes recognition \\
            \hspace{0.27cm}\textit{Using Top-down approach}
    \end{itemize}
            \begin{center}
                \includegraphics[width=0.4\textwidth]{face.jpg}
            \end{center}
    \begin{itemize}
        \item Cellphone detection \\
            \hspace{0.27cm}\textit{Using Bottoms-up approach}
    \end{itemize}
\end{frame}

\begin{frame}{Integration of tools}
    \begin{itemize}
        \item OpenCV to detect object on video stream
        \item \textbf{.DAT} file as face landmarks
        \item Caffe pre-trained model as cellphone detector
        \item All included on Python simple application
    \end{itemize}
\end{frame}

\begin{frame}{How does it works (1)}
    \begin{itemize}
        \item Attempt to detect objects on each frame:
            \begin{enumerate}
                \item A face
                \item Eyes
                \item A cellphone
            \end{enumerate}
        \item Face and eyes are compared with the \textbf{.DAT} file
        \item Computing eyelids distance to detect closed eyes
        \item Neural network to detect labeled foreign objects
    \end{itemize}
\end{frame}

\begin{frame}{How does it works (2)}
         Alert when:
            \begin{itemize}
                \item No face detected
                \item Eyes closed for too long
                \item Foreign object recognized as a cellphone
            \end{itemize}
\end{frame}

\begin{frame}{Usage}
    Python3 with options:
    \begin{itemize}
        \item \textbf{-s, --shape-predictor} Face landmarks (\textbf{.DAT} file)
        \item \textbf{-p, --prototxt} Caffe \textit{deploy} prototxt file
        \item \textbf{-m, --model} Caffe pre-trained model
        \item \textbf{-l, --label} Text file linking Caffe labels to text
        \item \textbf{-a, --alarm (optional)} Alarm \textbf{.WAV} file
        \item \textbf{-w, --webcam (optional)} Index of webcam on system
    \end{itemize}
\end{frame}

\begin{frame}{Thank you!}
    %\begin{itemize}
        \textbf{Do you have any questions?} \\
            \hspace{0.27cm}A video demo is ready!
    %\end{itemize}
\end{frame}

\begin{frame}{Credits}
    \begin{itemize}
        \item \url{https://link.springer.com/chapter/10.1007/978-3-642-21729-6_80}
        \item \url{https://www.pyimagesearch.com/2017/05/08/drowsiness-detection-opencv/}
        \item \url{https://www.pyimagesearch.com/2017/08/21/deep-learning-with-opencv/}
        \item \url{https://techcrunch.com/2016/11/13/wtf-is-computer-vision/}
       \end{itemize}
\end{frame}

\begin{frame}{Credits}
    \begin{itemize}
        \item \url{http://www.who.int/violence_injury_prevention/publications/road_traffic/distracted_driving_en.pdf}
        \item \url{https://www.cdc.gov/features/dsdrowsydriving/index.html}
        \item \url{https://opencv.org}
        \item \url{http://caffe.berkeleyvision.org}
    \end{itemize}
\end{frame}
\end{document}
